\section{Demo-Seite}
%Falls keine Seitenanzahl vorhanden ist  dann:\footcite[Vgl.][]{DemoQuelle}
Demo-Quelle \footcite[Vgl.][\printfield{pages}]{DemoQuelle}. 
In modernen Computern ist die \ac{CPU} ein wesentlicher Bestandteil. Die Taktfrequenz wird oft in \ac{KHZ} angegeben.

\begin{center}
    \begin{table}[h]
    \centering
    \begin{tabular}{|c|p{6cm}|}
        \hline
        \textbf{Datum} & \textbf{Aktivitäten} \\
        \hline
        Kebab & 7€ \\
        \hline
        Adana & \begin{itemize}
            \item \textbf{Groß}: 8€
            \item \textbf{Klein}: 6€
        \end{itemize} \\
        \hline
        Köfte & \begin{itemize}
            \item 5 Stück: 8€
            \item 2 Stück: 6€
        \end{itemize}\\
        \hline
        Mercimek Suppe & 3€ \\
        \hline
        Dönerteller & 15€ \\
        \hline
    \end{tabular}
    \caption{Beispiel Tabelle.}
    \label{tab:example}
    \end{table}
    \end{center}

Die Tabelle zeigt den Preis eines Dönertellers, dieser lässt sich wie folgt berechnen:
\begin{equation}
    15 = \sum_{n=1}^{10} \frac{n}{20} + \sum_{k=1}^{5} \frac{2k}{10} - \sum_{i=1}^{3} i
\end{equation}
\newpage
Der folgende Abschnitt könnte hilfreich für eine Ausarbeitung in der Informatik sein.
\begin{figure}[h]
\begin{lstlisting}
    def hello_world():
        print("Hello, World!")
\end{lstlisting}
\caption{Ausschnitt aus main.py.}
\label{fig:meincode}
\end{figure}
\\

Diese \ac{CPU} verarbeitet Daten in einem Takt von mehreren \ac{KHZ}.\\
Das kleine Kind war ganz fasziniert von den \acp{BluM}, und wollte unbedingt eine eigene \ac{BluM} als Spielzeug haben. 
Auf dem Kinderfest waren zahlreiche \acp{BluM} zu sehen, jede bunter und fröhlicher als die andere.\\

Erzwungene Kurzschreibweise: \acs{CPU}\\
Erzwungene Langschreibweise: \acl{CPU}