%Das Abkürzungsverzeichnis benutzt das package acronym
%hier ist die Doku https://www.ctan.org/pkg/acronym
%Nur im Text verwendete Abkürzungen werden angezeigt.
%2x Kompilieren für Aktualisierung des Abkürzungsverzeichnisses.
\section*{Abkürzungsverzeichnis}
\addcontentsline{toc}{section}{Abkürzungsverzeichnis}

\begin{acronym}[BluMä] % In den eckigen Klammern das längste Acronym schreiben, dies bestimmt dann das Spacing
  \acro{KHZ}{Kilohertz}
  \acro{CPU}{Central Processing Unit}
  % Falls eine Abkürzung in der Mehrzahl nicht einfach auf "s" endet, muss das speziell eingestellt werden.
  \acro{BluM}{Blaue lustige Maus}
  \acroplural{BluM}[BluMä]{Blaue lustige Mäuse}
\end{acronym}
\newpage