\section{Demo-Seite}
Auf dieser Seite befinden sich Umsetzungsbeispiele für häufig benötigte Elemente im Fließtext.\\
%Falls keine Seitenanzahl vorhanden ist  dann:\footcite[Vgl.][]{DemoQuelle}
Demo-Quelle \footcite[Vgl.][\printfield{pages}]{DemoQuelle}.

\begin{center}
    \begin{table}[h]
    \centering
    \begin{tabular}{|c|p{6cm}|}
        \hline
        \textbf{Datum} & \textbf{Aktivitäten} \\
        \hline
        Kebab & 7€ \\
        \hline
        Adana & \begin{itemize}
            \item \textbf{Groß}: 8€
            \item \textbf{Klein}: 6€
        \end{itemize} \\
        \hline
        Köfte & \begin{itemize}
            \item 5 Stück: 8€
            \item 2 Stück: 6€
        \end{itemize}\\
        \hline
        Mercimek Suppe & 3€ \\
        \hline
        Dönerteller & 15€ \\
        \hline
    \end{tabular}
    \captionwithfootnotemark{Beispiel Tabelle.}% Die (verpflichtende) Quellenangabe hierzu wird durch den \footcitetext-Befehl weiter unten gesetzt
    \label{tab:example}
    \end{table}
\end{center}
\footcitetext[Vgl.][\printfield{pages}]{DemoQuelle}\\ [-4em]

Die Tabelle zeigt den Preis eines Dönertellers, dieser lässt sich wie folgt berechnen:
\begin{equation}
    15 = \sum_{n=1}^{10} \frac{n}{20} + \sum_{k=1}^{5} \frac{2k}{10} - \sum_{i=1}^{3} i
\end{equation}

\newpage
Der folgende Abschnitt könnte hilfreich für eine Ausarbeitung in der Informatik sein.
\begin{figure}[h]
    \begin{lstlisting}
        def hello_world():
            print("Hello, World!")
    \end{lstlisting}
    \captionwithfootnotemark{Ausschnitt aus main.py.}% Die (verpflichtende) Quellenangabe hierzu wird durch den \footnotetext-Befehl weiter unten gesetzt
    \label{fig:meincode}
\end{figure}
\footnotetext{Quelle: Eigene Erstellung}

\subsection{Abkürzungen aus dem Verzeichnis}

Diese \ac{CPU} verarbeitet Daten in einem Takt von mehreren \ac{KHZ}.\\
Das kleine Kind war ganz fasziniert von den \acp{BluM}, und wollte unbedingt eine eigene \ac{BluM} als Spielzeug haben. 
Auf dem Kinderfest waren zahlreiche \acp{BluM} zu sehen, jede bunter und fröhlicher als die andere.\\

Erzwungene Kurzschreibweise: \acs{CPU}\\
Erzwungene Langschreibweise: \acl{CPU}

\newpage
\subsection{Zitierbeispiele}
\subsubsection{Beispiel für jeden Quelltyp}% "Vgl." steht nur in Fußnoten für indirekte (nicht wörtliche) Zitate.

Buch/Monografie\footcite[Vgl.][\printfield{pages}]{theisen2011}\\
Sammelwerk\footcite[Vgl.][\printfield{pages}]{maier2004}\\
Zeitschriften-/Journalartikel\footcite[Vgl.][\printfield{pages}]{chodorowreich2022loan}\\
Zeitungsartikel\footcite[Vgl.][\printfield{pages}]{dick2012neugierige}\\
Internet\footcite[Vgl.][]{capital2014}\\
Gesetztestext\footnote{Vgl. §433 Abs. 1 Satz 1 BGB}\nocite{bgb}\\
Gerichtsurteil\footcite[Vgl.][\printfield{pages}]{bverfgh1968}\\
öffentliches Dokument\footcite[Vgl.][\printfield{pages}]{eu2022access}\\
internes Dokument\footcite[Vgl.][\printfield{pages}]{abcorganigramm}\\% muss auf separatem Datenträger beigefügt werden
(unvollständige Quellenangaben)\footcite[Vgl.][]{blankmaier}

\subsubsection{Beispiel für verschiedene Arten von Zitaten}


\subsubsubsection{Beispiel für eine sub-sub-sub-Überschrift}% Diese Vorlage unterstützt (technisch und moralisch) nur vierfache Untergliederung
(siehe oben)