\section{Demo-Seite}
%Falls keine Seitenanzahl vorhanden ist  dann:\footcite[Vgl.][]{DemoQuelle}
Demo-Quelle \footcite[Vgl.][\printfield{pages}]{DemoQuelle}.

\begin{center}
    \begin{table}[h]
    \centering
    \begin{tabular}{|c|p{6cm}|}
        \hline
        \textbf{Datum} & \textbf{Aktivitäten} \\
        \hline
        Kebab & 7€ \\
        \hline
        Adana & \begin{itemize}
            \item \textbf{Groß}: 8€
            \item \textbf{Klein}: 6€
        \end{itemize} \\
        \hline
        Köfte & \begin{itemize}
            \item 5 Stück: 8€
            \item 2 Stück: 6€
        \end{itemize}\\
        \hline
        Mercimek Suppe & 3€ \\
        \hline
        Dönerteller & 15€ \\
        \hline
    \end{tabular}
    \captionwithfootnotemark{Beispiel Tabelle.}% Die (verpflichtende) Quellenangabe hierzu wird durch den \footcitetext-Befehl weiter unten gesetzt
    \label{tab:example}
    \end{table}
\end{center}
\footcitetext[Vgl.][\printfield{pages}]{DemoQuelle}

Die Tabelle zeigt den Preis eines Dönertellers, dieser lässt sich wie folgt berechnen:
\begin{equation}
    15 = \sum_{n=1}^{10} \frac{n}{20} + \sum_{k=1}^{5} \frac{2k}{10} - \sum_{i=1}^{3} i
\end{equation}

\newpage
Der folgende Abschnitt könnte hilfreich für eine Ausarbeitung in der Informatik sein.
\begin{figure}[h]
    \begin{lstlisting}
        def hello_world():
            print("Hello, World!")
    \end{lstlisting}
    \captionwithfootnotemark{Ausschnitt aus main.py.}% Die (verpflichtende) Quellenangabe hierzu wird durch den \footnotetext-Befehl weiter unten gesetzt
    \label{fig:meincode}
\end{figure}
\footnotetext{Quelle: Eigene Erstellung}
