\documentclass[a4paper,12pt]{article}

% Pakete
\usepackage[utf8]{inputenc}
\usepackage[T1]{fontenc}
\usepackage[ngerman]{babel}
\usepackage{csquotes}  % für korrekte Anführungszeichen
\usepackage[backend=biber, style=apa6]{biblatex}
\usepackage{csquotes}
\MakeOuterQuote{"}

% Literaturdatei einbinden
\addbibresource{quellen.bib}

\title{Einführung in \LaTeX}
\author{Anisa Beispiel}
\date{\today}

\begin{document}

\maketitle

\tableofcontents
\newpage

\section{Was ist \LaTeX?}
\LaTeX{} ist ein Textsatzsystem, das häufig in der Wissenschaft verwendet wird.
Es trennt Inhalt und Formatierung, was saubere und reproduzierbare Dokumente ermöglicht.

Ein Beispiel für eine Quelle ist \cite{DemoQuelle}.

\section{Fazit}
Mit \LaTeX{} lassen sich professionelle Dokumente effizient erstellen.

\newpage
\printbibliography

\end{document}
