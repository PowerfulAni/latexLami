\documentclass[a4paper,12pt]{article}

% Pakete
\usepackage[utf8]{inputenc}
\usepackage[T1]{fontenc}
\usepackage[ngerman]{babel}
\usepackage{csquotes}  % für korrekte Anführungszeichen
\usepackage[backend=biber, style=apa6]{biblatex}
\usepackage{csquotes}
\MakeOuterQuote{"}

% Literaturdatei einbinden
\addbibresource{quellen.bib}

\title{Einführung in \LaTeX}
\author{Anisa Beispiel}
\date{\today}

\begin{document}

\maketitle

\tableofcontents
\newpage

\section{Was ist \LaTeX?}
\LaTeX{} ist ein Textsatzsystem, das häufig in der Wissenschaft verwendet wird.
Es trennt Inhalt und Formatierung, was saubere und reproduzierbare Dokumente ermöglicht.\footcite{maier2004}

Ein Beispiel für eine Quelle ist \cite{DemoQuelle}.
\subsection{Entry-Typen}
\begin{table}[h!]
    \centering
    \begin{tabular}{ll}
    \textbf{Entry-Typ} & \textbf{Beschreibung} \\ \hline
    article & Artikel aus einer wissenschaftlichen Zeitschrift \\
    book & Eigenständiges Buch mit Verlag \\
    booklet & Gedrucktes Werk ohne Verlag oder Institution \\
    inbook & Kapitel oder Abschnitt aus einem Buch \\
    incollection & Beitrag in einem Sammelband mit eigenem Titel \\
    inproceedings & Artikel in Konferenz-Proceedings \\
    proceedings & Gesamter Konferenzband \\
    manual & Technische Dokumentation oder Handbuch \\
    mastersthesis & Masterarbeit (Universität oder Hochschule) \\
    phdthesis & Doktorarbeit (Dissertation) \\
    techreport & Technischer Bericht einer Institution \\
    unpublished & Noch nicht offiziell veröffentlichtes Werk \\
    misc & Alles, was in keine andere Kategorie passt \\
    online & Online-Quelle oder Webseite (BibLaTeX) \\
    report & Bericht, ähnlich wie techreport (BibLaTeX) \\
    thesis & Allgemeine Abschlussarbeit (BibLaTeX) \\
    collection & Sammelband mit mehreren Beiträgen (BibLaTeX) \\
    mvbook & Mehrbändiges Buch (BibLaTeX) \\
    mvcollection & Mehrbändiger Sammelband (BibLaTeX) \\
    mvproceedings & Mehrbändige Konferenzbände (BibLaTeX) \\
    inreference & Artikel in einem Nachschlagewerk (BibLaTeX) \\
    reference & Nachschlagewerk oder Enzyklopädie (BibLaTeX) \\
    dataset & Datensatz oder Datenveröffentlichung (BibLaTeX) \\
    software & Software oder Programmcode (BibLaTeX) \\
    \end{tabular}
    \caption{Übersicht der gängigen BibTeX- und BibLaTeX-Entrytypen}
    \end{table}
    
\section{Fazit}
Mit \LaTeX{} lassen sich professionelle Dokumente effizient erstellen.

\newpage
\printbibliography

\end{document}
