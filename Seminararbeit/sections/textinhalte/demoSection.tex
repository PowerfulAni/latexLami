\section{Demo-Seite}
Auf dieser Seite befinden sich Umsetzungsbeispiele für häufig benötigte Elemente im Fließtext.\\

\begin{center}
    \begin{table}[h]
    \centering
    \begin{tabular}{|c|p{6cm}|}
        \hline
        \textbf{Datum} & \textbf{Aktivitäten} \\
        \hline
        Kebab & 7€ \\
        \hline
        Adana & \begin{itemize}
            \item \textbf{Groß}: 8€
            \item \textbf{Klein}: 6€
        \end{itemize} \\
        \hline
        Köfte & \begin{itemize}
            \item 5 Stück: 8€
            \item 2 Stück: 6€
        \end{itemize}\\
        \hline
        Mercimek Suppe & 3€ \\
        \hline
        Dönerteller & 15€ \\
        \hline
    \end{tabular}
    \captionwithfootnotemark{Beispiel Tabelle.}% Die (verpflichtende) Quellenangabe hierzu wird durch den \footcitetext-Befehl weiter unten gesetzt
    \label{tab:example}
    \end{table}
\end{center}
\footcitetext[Vgl.][\printfield{pages}]{DemoQuelle}\\ [-4em]

Die Tabelle zeigt den Preis eines Dönertellers, dieser lässt sich wie folgt berechnen:
\begin{equation}
    15 = \sum_{n=1}^{10} \frac{n}{20} + \sum_{k=1}^{5} \frac{2k}{10} - \sum_{i=1}^{3} i
\end{equation}

\newpage
Der folgende Abschnitt könnte hilfreich für eine Ausarbeitung in der Informatik sein.
\begin{figure}[h]
    \begin{lstlisting}
        # say hi
        def hello_world():
            print("Hello, World!")
    \end{lstlisting}
    \captionwithfootnotemark{Ausschnitt aus main.py.}% Die (verpflichtende) Quellenangabe hierzu wird durch den \footnotetext-Befehl weiter unten gesetzt
    \label{fig:meincode}
\end{figure}
\footnotetext{Quelle: Eigene Erstellung}

\subsection{Zitierbeispiele}
\subsubsection{Beispiel für jeden Quelltyp}% "Vgl." steht nur in Fußnoten für indirekte (nicht wörtliche) Zitate.
% Wenn eine Quellenangabe sich auf einen ganzen Satz bzw. Absatz bezieht, wird die Fußnote erst am Satzende nach dem Punkt platziert.

Buch/Monografie\indirectcite{theisen2011}\\
Sammelwerk\indirectcite{maier2004}\\
Zeitschriften-/Journalartikel\indirectcite{chodorowreich2022loan}\\
Zeitungsartikel\indirectcite{dick2012neugierige}\\
Internet\indirectcite[ ]{capital2014}\\% falls es keine Seitenzahlen gibt, ein Leerzeichen als Seitenzahl-Argument übergeben (wird nicht in der PDF eingetragen)
Gesetzestext\footnote{Vgl. §433 Abs. 1 Satz 1 BGB}\nocite{bgb}\\
Gerichtsurteil\indirectcite{bverfgh1968}\\
öffentliches Dokument\indirectcite{eu2022access}\\
internes Dokument\indirectcite{abcorganigramm}\\% muss auf separatem Datenträger beigefügt werden. Am besten sammelt man schon frühzeitig alle Dokumente in einem Ordner.
(unvollständige Quellenangaben)\indirectcite{blankmaier}

\newpage
\subsubsection{Beispiele für verschiedene Arten von Zitaten}% Es genügt den Anforderungen, Kurzbelege zu setzen, die auf den Vollbeleg im Literaturverzeichnis verweisen.

indirektes Zitat (Seitenzahlen aus der .bib)\indirectcite{theisen2011}\\% Indirekte (sinngemäße) Zitate geben den Inhalt der Quelle in eigenen Worten wieder.
indirektes Zitat (individuelle Seitenzahlen)\indirectcite[23f.]{theisen2011}\\% f. (folgend) bezeichnet diese und die nächste Seite
direktes Zitat (Seitenzahlen aus der .bib)\directcite{theisen2011}\\% Direkte Zitate geben den Wortlaut der Quelle wieder. Sie müssen in Anführungszeichen gesetzt werden.
direktes Zitat (individuelle Seitenzahlen)\directcite[23f.]{theisen2011}

\subsubsubsection{Beispiel für eine sub-sub-sub-Überschrift}% Diese Vorlage unterstützt (technisch und moralisch) nur max. vierfache Untergliederung