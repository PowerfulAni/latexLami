\documentclass{article}

\usepackage{lmodern}
\usepackage[T1]{fontenc}
\usepackage[utf8]{inputenc}
\usepackage[ngerman]{babel}
\usepackage[most]{tcolorbox}
\usepackage{graphicx}
\usepackage{xcolor}
\usepackage[backend=biber, style=apa6]{biblatex}
\usepackage{array} % For vertical centering in tabular
\usepackage[margin=2cm]{geometry}
\usepackage{helvet} % or try [scaled]{avant}
\usepackage[colorlinks=true, linkcolor=blue, urlcolor=blue]{hyperref}
\usepackage[absolute,overlay]{textpos}
% display quotation marks ("") as their german counterparts („“)
\usepackage{csquotes}
\MakeOuterQuote{"}

\graphicspath{{images/}}

\renewcommand\familydefault{\sfdefault}

\definecolor{dbteal}{HTML}{1BA099} % Define the custom color

\title{\color{dbteal}\textbf{Installation von MiKTeX auf Windows}}
\date{}

% Define the highlighted paragraph environment
\newtcolorbox{highlighted}[1]{
  enhanced,
  boxrule=0pt,
  borderline west={1.5mm}{0pt}{#1},
  left=2mm, right=2mm, top=1mm, bottom=1mm,
  width=0.95\textwidth,
  % title={#1},
  % coltitle=black,   % Make the title text white (invisible on white background)
  % colback=white, 
  % colbacktitle=white,    % title background!
  % fonttitle=\bfseries  % Bold title font
}

\newcommand{\installstep}[2]{%
  \noindent
  \begin{tabular}{ >{\raggedright\arraybackslash}m{0.4\textwidth}  @{\hspace{0.06\textwidth}} m{0.48\textwidth} }
    #1 &
    \includegraphics[width=\linewidth]{#2}
  \end{tabular}
  \par
  \vspace{2em}
}

\newcommand{\warning}[1]{%
  \noindent
  \begin{highlighted}{red}
    #1
  \end{highlighted}
  \par
  \vspace{1.5em}
}

\newcommand{\notice}[1]{%
  \noindent
  \begin{highlighted}{dbteal}
    #1
  \end{highlighted}
  \par
  \vspace{1.5em}
}

\begin{document}
\noindent

\begin{textblock*}{3cm}(8.8cm,1.1cm)
  {\includegraphics[width=3cm]{logo.png}}
\end{textblock*}

\maketitle

\warning{
  Während der Installation darf man sich nicht im VPN bzw. im DB internen (W)LAN befinden, da sonst der MiKTeX-Paketserver (api2.miktex.org) nicht korrekt erreicht werden kann.
}

\notice{
  Für die Installation auf Windows zunächst den \href{https://miktex.org/download}{\underline{Installer}} herunterladen. Diesen öffnen und die Lizenzbedingungen akzeptieren. Dann wie folgt vorgehen:
}

\installstep{
  MiKTeX nur für den eigenen Windows User installieren. Admin-Berechtigungen sollten dafür nicht benötigt werden. Die Installation findet dann standardmäßig im Ordner /AppData/Local/Programs statt; das passt so.
}{scope}

\installstep{
  Bei den allgemeinen Einstellungen die Optionen "A4" und "Yes" auswählen. In den folgenden Ansichten die Installation abschließen.
}{preferences}

\begin{center}
  \textit{Weiter auf Seite 2}
\end{center}
\pagebreak

\installstep{
  Nach der Installation: nach der "MiKTeX Console" suchen und diese öffnen. In der Konsole den Abschnitt links "Updates" auswählen.
}{console-search}

\installstep{
  Nach Updates suchen und (falls vorhanden) diese Updates mit "jetzt updaten" installieren.
}{console-updates}

\notice{
  Jetzt sollte man mit LaTeX startklar sein. Ob die Installation funktioniert hat, kann man testen, indem man in der Windows Powershell (bzw. CMD) den Befehl \textit{where pdflatex} ausführt; es sollte der Installationspfad unter /AppData/ angezeigt werden.
}

\warning{
  Sollte es beim Updaten zu Fehlern kommen, liegt es wahrscheinlich daran, dass man sich im VPN bzw. im DB internen (W)LAN befindet.
}

\warning{
  Bei einer Re-Installation von MiKTeX kann es zu dem Fehler "MiKTeX Console is already running"\\ kommen, falls die Deinstallation nicht vollständig ausgeführt wurde. In diesem Fall zum Ordner C:/Users/NutzerName navigieren und die Datei miktex-console.lock löschen.
}

\end{document}
